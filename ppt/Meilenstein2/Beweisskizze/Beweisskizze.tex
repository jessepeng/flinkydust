\documentclass{article}		%sonst [final]
%%%%
%PACKAGES
%%%%
\usepackage[utf8]{inputenc}		% Umlaute
\usepackage{lastpage}						% \pageref{LastPage}
\usepackage{fancyhdr}						% alternative header
\usepackage{geometry}
\usepackage{amsmath}
\usepackage{amsfonts}
\usepackage{amssymb}
\usepackage{wasysym}
\usepackage{graphicx}
\usepackage{epstopdf}
\usepackage[mathscr]{euscript}
\usepackage{algorithm2e}
\geometry{a4paper,left=5mm,right=5mm,top=25mm,bottom=35mm}
\usepackage{listings}\lstset{numbers=left, numberstyle=\tiny, numbersep=5pt} \lstset{language=Perl} 
%%%%
%\title{Erstsemestertutorium (Latex)}
%%%%
%FANCYHDR SETTINGS
%%%%
%%%%
\pagestyle{fancy} 	% Seitenstil
% \fancyhf{}				% Kopf- und Fußzeilenfelder bereinigen
% \fancypagestyle		% redefinieren des pagestyles
%%%%
%HEAD
%%%%
\setlength{\headheight}{56pt}		% Größe des Header
\fancyhead[L]{Beweisskizze}		% Kopfzeile links
\fancyhead[C]{}		% zentrierte Kopfzeile%Pfadangabe zum Logo, ACHTUNG auf die  "/" %
\fancyhead[R]{\date}		% Kopfzeile rechts
\renewcommand{\headrulewidth}{0.4pt}		% Breite obere Trennlinie
%------------------------------------------------------------
% oder \fancyhead[XX]{}
%%%%
%FOOT
%%%%
\lfoot{}
\cfoot{}
\rfoot{Seite \thepage /\pageref{LastPage}}
\renewcommand{\footrulewidth}{0.4pt} %untere Trennlinie
%------------------------------------------------------------
%oder \fancyfoot[XX]{}
%
%Definition von X:
%----------------
% E: Even page 		O: Odd page				(gerade/ungerade)
% L: Left field		C: Center field		R: Right field
% H: Header 			F: Footer
%%%%
\parindent 0pt		% keine Einrückung



\begin{document}

\texttt{Aufgabe: Zeige, dass ein Datentyp mit den Operatoren $\sigma$ (Selektion), $\pi$ (Projektion), $\gamma$ (Aggregation) und \\
$\times$ (Kartesisches Produkt) die folgenden Tasks unterstützt...}\\

\textbf{1. Identifizieren}:\\
$A=\{a_1,...,a_n\}$ sei eine Menge von n Objekten. Identifiziere Objekte in A die eine bestimmte Bedingung erfüllen.\\
\smallskip

\textit{Definition $\sigma$}: Seien $D_1,...,D_n$ Domänen und sei $R \subseteq D_1 \times ... \times D_n $ mit $R \{A_1:D_1,...,A_n:D_n\}$ eine n-stellige Relation auf diesen Domänen. Sei $c$ eine Selektionsbedingung, d. h. ein Boolscher Ausdruck aus Attributen ($A_1,...,A_n$), Operatoren ($= , \neq , \geq , \leq , <, > $) und logischen Junktoren ($\wedge , \vee $). Dann ist die Selektion wie folgt definiert: 
\[ \sigma_c (R) := \{ \mu: (c \left[ \mu \right] = true) \wedge (\mu \in R)\} \] wobei $\mu$ die Tupel der Relation sind.\\
\smallskip

Die Datenstruktur enthält eine Menge A von Objekten. Die Objekte sind gleichförmige Elemente der Extension einer Relation, d.h. jedes Objekt ist ein Tupel einer bestimmten Relation R. Die Datenstruktur unterstützt weiterhin den Operator Selektion. Die Selektion $\sigma$ ist äquivalent zu dem Task "Identifizieren", soweit sich die gefordeten Bedingungen als Boolscher Ausdruck beschreiben lassen. Somit lässt sich der Task "identifizieren" durch den Operator $\sigma$ realisieren.

\bigskip

\bigskip

\textbf{2. Vergleichen}:\\
 $A=\{a_1,...,a_n\}$ sei eine Menge von n Objekten und $C^k = A \times_1 ... \times_k A$ eine beliebige Relation. Vergleiche Objekte $\{a_1,...,a_k\}$ in A um geordnete Paare $a_{\pi (1)}Ca_{\pi (2)}C...Ca_{\pi (k-1)}Ca_{\pi (k)}$ zu erkennen die $C^k$ erfüllen  ($\pi$ ist eine valide Permutation der Indizes). \\

\bigskip

\textbf{3. Merkmale erkennen}:\\
 $A=\{a_1,...,a_n\}$ sei eine Menge von n Objekten und $F_l$ eine Familie von Funktionen. Erkenne alle Untermengen $\{a_1,...,a_k\}$, die eine Funktion $F \in F_l$ zu \textit{true} auswerten.\\


\end{document}